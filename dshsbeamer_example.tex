% An inofficial beamer theme for the German Sport University Cologne
% This LaTeX code can be redistributed and/or modified under the terms of the GNU Public License, version 3.
%
% Magnus Metz
% m.metz@dshs-koeln.de
%
% Compiled with XeLaTeX
% Dependencies:
%   ITC Officina Sans LT font 
%
\documentclass{beamer}

\usepackage{graphicx} % graphics
\usepackage{import} % import external graphics written in tikz
\usepackage{tikz} % needed to show graphics written in tikz
\usepackage{hyperref} 
\usepackage{booktabs, caption} % table styling

% suppress navigation bar
\beamertemplatenavigationsymbolsempty

\mode<presentation>
{
  \usetheme{dshsbeamer}
  \setbeamercovered{transparent}
  \setbeamertemplate{items}[circle]
}

% set fonts
\usepackage{fontspec}
%\setsansfont[Mapping=tex-text, BoldFont={ITC Officina Sans LT Bold}, ItalicFont={ITC Officina Sans LT Book Italic}]{ITC Officina Sans LT Book} %This is the official font of the German Sport University Cologne. You need to have it installed on your computer in order to use it!
\setbeamerfont{frametitle}{size=\LARGE,series=\bfseries}

% color definitions
\usepackage{color}
\definecolor{spohoblue}{RGB}{0, 83, 146}
\definecolor{black}{RGB}{0, 0, 0}

% caption styling
\DeclareCaptionFont{black}{\color{black}}
\DeclareCaptionFont{spohoblue}{\color{spohoblue}}
\captionsetup{labelfont={spohoblue},textfont=black}

% adds reference to bottom right of corner of a slide
\usepackage[absolute,overlay]{textpos} % text references in slide corners
\newcommand\textref[1]{%
  \begin{textblock*}{\paperwidth}(0pt,0.99\textheight)
  \raggedleft \tiny{\emph{#1}}\hspace{.5em}
  \end{textblock*}}

\usepackage{url}

% title slide definition
\title[Short title]{An unofficial beamer template for the German Sport University Cologne}
\author[Short author]{Magnus Metz}
\institute[]
{Institute of Sport Economics and Sport Management \\
German Sport University Cologne \\
}

\date{\today}

% logo on titlepage:
%\titlegraphic{\includegraphics[width=3.5cm,height=0.8cm]{logo_dshs.jpg}}

%--------------------------------------------------------------------
%                           Document
%--------------------------------------------------------------------

\begin{document}
\urlstyle{same}

\begin{frame}[plain]
 \titlepage
\end{frame}

\begin{frame}{Contents}
\tableofcontents
\end{frame}


\section{Some basic slides}
\subsection{An impression}
\begin{frame}{First slide for}
    
    \vspace{1cm} % generate some space between title and content
    \begin{table}[h]
    \centering
    \begin{tabular}{lcccc} \bottomrule[2pt]
        Name & Time & Mean & M.P. (K) & IE (J) \\ \bottomrule 
        Felix & 10.9 & 4.00 & 1 & 3.78 \\
        Caro & 11.0 & 12.01 & 773 & 3.94 \\
        Mathew & 10.7 & 74.92 & 1090 & 1.48 \\
        Alex & 9.9 & 196.96 & 1337 & 1.48 \\
        Markus & 11.5 & 58.93 & 1495 & 1.26 \\
    \bottomrule[2pt]
    \end{tabular}
    \caption{some random data in a dummy table}
    \end{table}

\vspace{-0.6cm} % compact spacing between table and text

    \begin{columns}[t]
    \column{4.5cm}
    \begin{block}{Trace rare earth metals:}
    \begin{itemize}
        \item{Ytterbium}
        \item{Praseodymium}
        \item{Neodymium}
    \end{itemize}
    \end{block}
    \column{4.5cm}
    \begin{block}{Obtaining Neodymium:}
        \begin{enumerate}
        \item First item
        \item Second item
        \end{enumerate}
    \end{block}
    \end{columns}

    \textref{Write references here}

\end{frame}



\section{Including graphics}
\begin{frame}{Including graphics}
\begin{figure}
\begin{center}
\import{}{dia_graphic}
\caption{A tikz graphic drawn with DIA}
\end{center}
\end{figure}
\end{frame}


\begin{frame}[plain]
\begin{center}
Thank you for your attention!
\end{center}
\end{frame}

\end{document}
